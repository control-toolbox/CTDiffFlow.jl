% -----------------------------------------------------------------------------------------------------------------------------------------
% Tag a gauche ou a droite : utile pour la numerotation des equations
%\makeatletter
%\newcommand{\reqnomode}{\tagsleft@false\let\veqno\@@eqno}
%\newcommand{\leqnomode}{\tagsleft@true\let\veqno\@@leqno}
%\makeatother

%table des mati\`eres et compteurs de titres
\setcounter{secnumdepth}{3}
\setcounter{tocdepth}{2}

%\allowdisplaybreaks %Autorises latex à casser tous les blocs d'equations du style align ou equation.
\usepackage{pdfsync}

%Draft : to comment at the end
%\usepackage{showlabels}

%Les enumerations
\usepackage{enumerate}

%Les images
\usepackage{graphicx}
\graphicspath{{figures/pdf/}{figures/}}
\newcommand{\tfig}{0.33}
\newcommand{\tfigGrand}{0.39}
\newcommand{\tfigPetit}{0.3}
\newcommand{\tfigTrois}{0.3}

%Les dessins
\usepackage{circuitikz}
\usetikzlibrary{arrows,shapes,backgrounds,patterns}
\usepackage{pifont}
\usepackage{tikzgraphicx}
\tikzstyle{every picture}+=[remember picture]
\tikzstyle{na} = [baseline=-.5ex]
\makeatletter
\newcommand{\gettikzxy}[3]{%
  \tikz@scan@one@point\pgfutil@firstofone#1\relax
  \edef#2{\the\pgf@x}%
  \edef#3{\the\pgf@y}%
}
\makeatother

%Les couleurs
%\usepackage{xcolor}
%%\definecolor{red}{rgb}{0.6,0.04,0.04}
%\definecolor{green}{rgb}{0.06,0.5,0.06}
%\definecolor{blue}{rgb}{0.06,0.06,0.6}
%\definecolor{red}{rgb}{0.6,0.04,0.04}
%\definecolor{grey}{rgb}{0.6,0.6,0.6}
%\definecolor{black}{rgb}{0,0,0}
%\definecolor{rose}{RGB}{254,40,255}
%\newcommand{\cblue}[1]{{\color{blue} #1}}
%\newcommand{\cred}[1]{{\color{red} #1}}
%\newcommand{\cgreen}[1]{{\color{green} #1}}
%\newcommand{\crose}[1]{{\color{rose} #1}}
%\def\empha{\cblue}
%\def\emphb{\cred}
%\def\emphc{\cgreen}
%\def\emphd{\crose}
%\def\ea{\empha}
%\def\eb{\emphb}
%\def\ec{\emphc}
%\def\ed{\emphd}

%La commande pour la todo_list
\newcommand{\todo}[1]{\noindent {\color{blue} \it Note: #1}}

%Quelques commandes
\renewcommand\ie{\textit{i.e. }}

%Tableau : \'epaisseur des lignes
\usepackage{booktabs}
\newcommand{\smallhrule}{\specialrule{0.02em}{0.3em}{0.3em}}
\newcommand{\medhrule}{\specialrule{0.05em}{0.3em}{0.3em}}
\newcommand{\bighrule}{\specialrule{0.1em}{0.3em}{0.5em}}

%Les diff\'erents ensembles math\'ematiques
\usepackage{amsmath}
\makeatletter
\makeatother

\usepackage{amssymb}
\usepackage{mathrsfs}
\usepackage{mathtools}
\usepackage{xstring}

%Commandes persos
\newcommand\cste{a}
%\usepackage[f]{esvect}
%\usepackage{kpfonts}
\newcommand\vvec{\overrightarrow}
\newcommand{\frp}[2]{
    \frac{\partial {#1}}{\partial {#2}}
}
\newcommand\xt{\tilde{x}}
\newcommand\ft{\tilde{f}}
\newcommand\gt{\tilde{g}}
\newcommand\ct{\tilde{c}}
\newcommand\Ft{\tilde{F}}
\newcommand\Ht{\tilde{H}}
\newcommand\Et{\tilde{E}}
\newcommand\Jt{\tilde{J}}
\newcommand\At{\tilde{A}}
\newcommand\Pt{\tilde{\Phi}}
\newcommand\lsol{\bar{\lambda}}
\newcommand\ysol{\bar{y}}
\newcommand\xsol{\bar{x}}
\newcommand\usol{\bar{u}}
\newcommand\zsol{\bar{z}}
\newcommand\psol{\bar{p}}
\newcommand\tfsol{\bar{t}_f}
\newcommand\tsol{\bar{t}\hspace{0.1em}}
\newcommand\etasol{\bar{\eta}}
\newcommand\musol{\bar{\mu}}
\DeclareMathOperator{\ad}{ad}
\DeclareMathOperator{\rank}{rank}
\DeclareMathOperator{\lie}{Lie}
\newcommand{\bocop}{\texttt{Bocop}}
\newcommand{\hybrj}{\texttt{hybrj}}
\newcommand{\dopri}{\texttt{dopri5}}
\newcommand{\dop}{\texttt{dop853}}
\newcommand{\radau}{\texttt{radau}}
\newcommand{\minpack}{\texttt{minpack}}
\newcommand{\tapenade}{\texttt{tapenade}}
\newcommand{\expmap}[3]{\exp({#2 #3}) (#1)}
\newcommand{\Hom}{h}

\newcommand{\imax}{i_\mathrm{max}}
\newcommand{\vmax}{v_\mathrm{max}}
\newcommand{\af}{\alpha_f}

%Les packages
\usepackage{stmaryrd}

%Intervalle
\newcommand{\intervalle}[4]{\mathopen{#1}#2
                                \mathclose{}\mathpunct{},#3
                                \mathclose{#4}}
\newcommand{\intervalleff}[2]{\intervalle{[}{#1}{#2}{]}}
\newcommand{\intervalleof}[2]{\intervalle{(}{#1}{#2}{]}}
\newcommand{\intervallefo}[2]{\intervalle{[}{#1}{#2}{)}}
\newcommand{\intervalleoo}[2]{\intervalle{(}{#1}{#2}{)}}
\newcommand{\intervalleentier}[2]{\intervalle{\llbracket}{#1}{#2}{\rrbracket}}

%Ensembles de nombres
\newcommand{\nbSet}[1]{\mathbb{#1}}
\newcommand{\setPositive}{\text{\bf{\tiny+}}}
\newcommand{\setNegative}{\mathbb{\tiny-}}
\newcommand{\setStar}{\text{*}}
%\newcommand{\N}{\nbSet{N}}
%\newcommand{\Z}{\nbSet{Z}}
%\newcommand{\Q}{\nbSet{Q}}
%\newcommand{\R}{\nbSet{R}}
%\newcommand{\C}{\nbSet{C}}
%\newcommand{\R}{\xR}

\newcommand{\setDeco}[2]{
    \IfEqCase{#2}{
        {s}{\nbSet{#1}^{\setStar}}
        {n}{\nbSet{#1}^{\phantom{\setStar}}_{\setNegative}}
        {p}{\nbSet{#1}^{\phantom{\setStar}}_{\setPositive}}
        {sn}{\nbSet{#1}^{\setStar}_{\setNegative}}
        {sp}{\nbSet{#1}^{\setStar}_{\setPositive}}
    }
}
\newcommand{\Ns}{ \ensuremath{\setDeco{N}{s}} }
\newcommand{\Rn}{ \ensuremath{\setDeco{R}{n}} }
\newcommand{\Rp}{ \ensuremath{\setDeco{R}{p}} }
\newcommand{\Rs}{ \ensuremath{\setDeco{R}{s}} }
\newcommand{\Rsp}{ \ensuremath{\setDeco{R}{sp}} }
\newcommand{\Rsn}{ \ensuremath{\setDeco{R}{sn}} }

\newcommand{\Ucal}{\mathcal{U}}
\newcommand{\Acal}{\mathcal{A}}
\newcommand{\Lcal}{\mathcal{L}}
\newcommand{\Ccal}{\mathcal{C}}
\newcommand{\Tcal}{\mathcal{T}}

%Elements differentiel
%\newcommand{\diff}{\mathop{}\mathopen{}\mathrm{d}}
\newcommand{\Diff}{ {\rm D} }
%\newcommand{\diff}{\,{\rm d}}

%Valeur absolue et norme
%\newcommand{\abs}[1]{\lvert#1\rvert} 
%\newcommand{\absStyle}[1]{\left\lvert#1\right\rvert}
%\newcommand{\norme}[1]{\lVert#1\rVert}
%\newcommand{\normeStyle}[1]{\left\lVert#1\right\rVert}

%Petit o et grand O
\newcommand{\petito}[1]{o\mathopen{}\left(#1\right)}
\newcommand{\grandO}[1]{O\mathopen{}\left(#1\right)}

%Ensemble des .. tels que ..
\newcommand{\enstq}[2]{\left\{#1\mathrel{}\middle|\mathrel{}#2\right\}}

%Produit scalaire
\newcommand{\prodscal}[2]{\left\langle#1,#2\right\rangle}

%L'arc de cercle pour les champs de Jacobi
%\usepackage{yhmath}

%---
\makeatletter
\let\save@mathaccent\mathaccent
\newcommand*\if@single[3]{%
  \setbox0\hbox{${\mathaccent"0362{#1}}^H$}%
  \setbox2\hbox{${\mathaccent"0362{\kern0pt#1}}^H$}%
  \ifdim\ht0=\ht2 #3\else #2\fi
  }
%The bar will be moved to the right by a half of \macc@kerna, which is computed by amsmath:
\newcommand*\rel@kern[1]{\kern#1\dimexpr\macc@kerna}
%If there's a superscript following the bar, then no negative kern may follow the bar;
%an additional {} makes sure that the superscript is high enough in this case:
\newcommand*\widebar[1]{\@ifnextchar^{{\wide@bar{#1}{0}}}{\wide@bar{#1}{1}}}
%Use a separate algorithm for single symbols:
\newcommand*\wide@bar[2]{\if@single{#1}{\wide@bar@{#1}{#2}{1}}{\wide@bar@{#1}{#2}{2}}}
\newcommand*\wide@bar@[3]{%
  \begingroup
  \def\mathaccent##1##2{%
%Enable nesting of accents:
    \let\mathaccent\save@mathaccent
%If there's more than a single symbol, use the first character instead (see below):
    \if#32 \let\macc@nucleus\first@char \fi
%Determine the italic correction:
    \setbox\z@\hbox{$\macc@style{\macc@nucleus}_{}$}%
    \setbox\tw@\hbox{$\macc@style{\macc@nucleus}{}_{}$}%
    \dimen@\wd\tw@
    \advance\dimen@-\wd\z@
%Now \dimen@ is the italic correction of the symbol.
    \divide\dimen@ 3
    \@tempdima\wd\tw@
    \advance\@tempdima-\scriptspace
%Now \@tempdima is the width of the symbol.
    \divide\@tempdima 10
    \advance\dimen@-\@tempdima
%Now \dimen@ = (italic correction / 3) - (Breite / 10)
    \ifdim\dimen@>\z@ \dimen@0pt\fi
%The bar will be shortened in the case \dimen@<0 !
    \rel@kern{0.6}\kern-\dimen@
    \if#31
      \overline{\rel@kern{-0.6}\kern\dimen@\macc@nucleus\rel@kern{0.4}\kern\dimen@}%
      \advance\dimen@0.4\dimexpr\macc@kerna
%Place the combined final kern (-\dimen@) if it is >0 or if a superscript follows:
      \let\final@kern#2%
      \ifdim\dimen@<\z@ \let\final@kern1\fi
      \if\final@kern1 \kern-\dimen@\fi
    \else
      \overline{\rel@kern{-0.6}\kern\dimen@#1}%
    \fi
  }%
  \macc@depth\@ne
  \let\math@bgroup\@empty \let\math@egroup\macc@set@skewchar
  \mathsurround\z@ \frozen@everymath{\mathgroup\macc@group\relax}%
  \macc@set@skewchar\relax
  \let\mathaccentV\macc@nested@a
%The following initialises \macc@kerna and calls \mathaccent:
  \if#31
    \macc@nested@a\relax111{#1}%
  \else
%If the argument consists of more than one symbol, and if the first token is
%a letter, use that letter for the computations:
    \def\gobble@till@marker##1\endmarker{}%
    \futurelet\first@char\gobble@till@marker#1\endmarker
    \ifcat\noexpand\first@char A\else
      \def\first@char{}%
    \fi
    \macc@nested@a\relax111{\first@char}%
  \fi
  \endgroup
}
\makeatother

%................ ................ ................ ................  ................
%----  (D) MACROS FOR MATHEMATICS 
% From COCV
%................ ................ ................ ................  ................
\def\zQuaternion{\mathbb{H}} \def\xC{\mathbb{C}} \def\xR{\mathbb{R}}
\def\zQ{\mathbb{Q}} \def\xZ{\mathbb{Z}} \def\xN{\mathbb{N}}
\def\zP{\mathbb{P}} \def\xA{\mathbb{A}}
%--
\def\zCzero{{\rm C}^{0}}
\def\zCone{{\rm C}^{1}} 
\def\zCtwo{{\rm C}^{2}} 
\def\zCinfty{{\rm C}^{\infty}} 
\def\zCn#1{{\rm C}^#1}
%--
\def\zHzero{{\rm H}^{0}}
\def\zHone{{\rm H}^{1}}
\def\zHtwo{{\rm H}^{2}} 
\def\zHinfty{{\rm H}^{\infty}}
\def\zHn#1{{\rm H}^#1}
%
\def\zWzero{{\rm W}^{0}}
\def\zWone{{\rm W}^{1}}
\def\zWtwo{{\rm W}^{2}} 
\def\zWinfty{{\rm W}^{\infty}}
\def\zWn#1{{\rm W}^#1}
%
\def\zLzero{{\rm L}^{0}}
\def\zLone{{\rm L}^{1}}
\def\zLtwo{{\rm L}^{2}} 
\def\zLinfty{{\rm L}^{\infty}} 
\def\zLn#1{{\rm L}^#1}
%-- 
\def\zDif{{\rm D}}
\def\zdif{\,{\rm d}}
%-- 
\def\zdrv#1#2{\frac{{\rm d}#1}{{\rm d}#2}}%  "d#1 over d#2"
\def\zDrv#1#2{\frac{{\rm d}}{{\rm d}#2}#1}%  "d   over d#2  #1"
%--
\def\zker{\mathop{\rm ker\,}\nolimits}
\def\zcoker{\mathop{\rm coker\,}\nolimits}
\def\zim{\mathop{\rm im\,}\nolimits}
\def\zcoim{\mathop{\rm coim\,}\nolimits}
\def\zdim{\mathop{\rm dim\,}\nolimits}
\def\zcodim{\mathop{\rm codim\,}\nolimits}
\def\ztr{\mathop{\rm tr\,}\nolimits}
\def\zHom{\mathop{\rm Hom\,}\nolimits}
\def\zExt{\mathop{\rm Ext\,}\nolimits}
\def\zTor{\mathop{\rm Tor\,}\nolimits}
%--
\def\zGL{\mathop{\rm GL\,}\nolimits}
\def\zSL{\mathop{\rm SL\,}\nolimits}
\def\zPSL{\mathop{\rm PSL\,}\nolimits}
\def\zSO{\mathop{\rm SO\,}\nolimits}
\def\zSU{\mathop{\rm SU\,}\nolimits}
%

%Epsilon, phi, theta, etc.
%\newcommand{\veps}{\varepsilon}

\DeclareMathOperator{\acos}{acos}
\DeclareMathOperator{\trace}{trace}

\newcommand{\arc}{\gamma}

